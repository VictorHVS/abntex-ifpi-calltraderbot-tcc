% ----------------------------------------------------------
% Conclusão
% ----------------------------------------------------------
\chapter{Considerações Finais}

% Este trabalho apresentou a ferramenta Ajuda.Ai, uma ferramenta para captação de recursos em modalidade \emph{Crowdfunding} para ONGs e Instituições sem fins lucrativos. Para isso, foi feito um estudo sobre o fenômeno do \emph{crowdsourcing} na era digital e do \emph{crowdfunding} que utiliza de \emph{crowdsourcing} para arrecadar recursos financeiros. Além disso, o sistema desenvolvido para atender essa necessidade foi desenvolvido com boas práticas da indústria, utilizando técnicas e ferramentas de ponta como REST Web e CDI.

% Considera-se que a solução alcança seu objetivo de ser uma alternativa de baixo custo, de fácil uso e flexível por se utilizar de uma arquitetura moderna que permite interfaces mais elaboradas e consequentemente adequadas aos usuários do serviço. A interface padrão disponibilizada junto ao projeto, uma página web no modelo SPA, implementa acesso às funcionalidades da API e tenta, através da aplicação de técnicas de \emph{Call-to-Action}, melhorar a performance das doações às instituições. Sua arquitetura permite que instituições criem, por exemplo, aplicativos para dispositivos móveis que utilizam o Ajuda.Ai como suporte para funcionamento da arrecadação de doações.

% Por fim, o trabalho também contribui com a propagação do conhecimento em modelagem, utilização de ferramentas avançadas e boas práticas para implementação do suporte a diferentes \emph{Gateways} de pagamento, além de apresentar uma arquitetura moderna e adequada a grande variedade de meios de acesso a páginas web disponíveis.

% O código-fonte da ferramenta está disponível no endereço https://github.com/g0dkar/ajuda-ai, está disponível para acesso no endereço https://ajuda.ai e a API pode ser acessada através do \emph{endpoint} https://api.ajuda.ai/v1.