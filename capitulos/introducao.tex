% ----------------------------------------------------------
% Introdução
% ----------------------------------------------------------
\chapter{Introdução}

Em 2015 a taxa, considerada taxa básica de juros, Selic (Sistema Especial de Liquidação de Custódia) havia chegado a 14,25\% e permaneceu nesse nível por 15 meses, até que em outubro de 2016 a taxa começou a baixar gradualmente e em fevereiro de 2020, o Comitê de Política Monetária (Copom) promoveu novo corte, chegando a baixa história de 4,25\%.

Com esta redução expressiva da taxa Selic, muitos investimentos de renda fixa tem sua lucratividade afetada por possuírem rentabilidades diretamente associada à taxa básica de juros, tais como: caderneta de poupança, títulos públicos, CDI (Certificado de Depósito Interbancário), LCI (Letras de Crédito Imobiliário), LCA (Letras de Crédito do Agronegócio), LC (Letras de Câmbio), CDB (Certificado de Depósito Bancário), CRI (Certificado de Recebíveis Imobiliários), CRA (Certificado de Recebíveis do Agronegócio) e Debêntures. 

Sobre a poupança, vale ressaltar que por lei toda vez que a Selic for igual ou inferior a 8,5\%, a remuneração da aplicação passa a ser de 70\% da Selic acrescida da Taxa Referencial (TR), que atualmente está em zero. Com isso, o rendimento ficará abaixo de 3\%, inferior à inflação de 3,4\% projetada para o ano, segundo relatório focus. Desta forma, a modalidade de investimento mais utilizada no Brasil, teve retirada histórica em janeiro de 2020 ultrapassando R\$12 bilhões como divulgado pelo Banco Central.

Além disso, a redução da taxa, cria um ambiente favorável ao crescimento privado e por sua vez a renda variável. Tal cenário é evidenciado com a grande onda de novos ingressantes na bolsa de valores brasileira, conhecida como B3 (Brasil Bolsa Balcão). Que de acordo com dados divulgados em 2019, 867.742 mil novas pessoas entraram no mercado de renda variável, alcançando um total de 1.830.745 pessoas, três vezes mais investidores que há 10 anos.

Segundo relatório divulgado pela Associação Brasileira das Entidades dos Mercados Financeiro e de Capitais (ANBIMA) 70\% dos brasileiros vão pessoalmente até o banco para aplicar efetivamente seu dinheiro. Mas há quem prefira usar a tecnologia a seu favor: quase metade das pessoas investe pelo site do banco (14\%) ou da corretora (4\%) ou até pelo aplicativo do banco (29\%) ou da corretora (4\%).

Diante do cenário de crescimento e popularidade do mercado de ações no Brasil e no mundo, além da possibilidade de utilizar tecnologias móveis como aplicativos de negociação para interação simples e direta com o mercado de ações, uma ferramenta para estudo e prática na compra e venda de ativos financeiros se faz oportuna.

\section{Objetivos}
\subsection{Objetivo Geral}
O objetivo geral deste trabalho é prover uma aplicação Android de código aberto que viabilize a compra e venda de ativos financeiros em um ambiente emulado, que trabalhe com informações reais em tempo real, gratuito e sem riscos de prejuízo.

\subsection{Objetivos Específicos}
\begin{lista}
  \item Planejar e desenvolver uma aplicação Android com as funcionalidades necessárias para operações de compra e venda de ativos financeiros;
  \item Disponibilizar uma versão do aplicativo para download (Google Play).
\end{lista}