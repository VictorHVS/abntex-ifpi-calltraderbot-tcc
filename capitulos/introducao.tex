% ----------------------------------------------------------
% Introdução
% ----------------------------------------------------------
\chapter{Introdução}

A humanidade utilizou diversos instrumentos tangíveis como meio de troca durante a sua evolução, de conchas e sal até pedaços de papel com os quais somos familiares. Contudo, no decorrer da história, o desenvolvimento tecnológico eliminou, em um processo de ordem espontânea, grande parte do meio físico de troca, tornando-o eletrônica e preservando apenas a parte mais importante: a informação e o valor atribuído a este instrumento pela sociedade. 

O conceito destes instrumentos de troca, as moedas, é determinado pela Constituição Brasileira no artigo 21 do inciso VII e a Lei. 9.069, de 29 de junho de 1995, que prediz as condições para a sua emissão demostrando que a moeda é instituída por imposição legal. Em 2013, com o surgimento e ascensão das moedas eletrônicas, a lei nº 12.865, de 9 de outubro de 2013, tratou diretamente de disciplina-las conceituando-as como “recursos armazenados em dispositivo ou sistema eletrônico que permitem aos usuário final efetuar transação de pagamento”.

Seguindo com o avanço das tecnologias e da programação, na década passada um novo conceito de moeda surgiu, diferente das moedas eletrônicas, com o intuito de revolucionar as trocas monetárias de todo o mundo sem a intervenção de governos ou de instituições particulares com o propósito de lucrar com esse movimento. Em 31 de Outubro de 2008, é publicado um \textit{paper} intitulado "Bitcoin: Um Sistema de Dinheiro Eletrônico Ponto-a-Ponto"  ("Bitcoin: A Peer-to-Peer Electronic Cash System" em inglês), escrito por um autor sob o pseudônimo de Satoshi Nakamoto, dando inicio as criptomoedas.

Em pouco tempo, a confiança no sistema divulgado começa a crescer e a primeira compra com Bitcoin é feita: Duas pizzas são compradas por 10.000 Bitcoin, inaugurando o Bitcoin como um meio de pagamento efetivo. Desde então muita coisa mudou, o Bitcoin passou a ser aceito mundialmente por empresas, comércios, prestadores de serviços e muitas outras criptomoedas foram criadas a partir do Bitcoin com o propósito de trazer melhorias. 

A tecnologia por trás de seu funcionamento, a Blockchain, está revolucionando inúmeros setores além do financeiro, sendo estudado por gigantes tecnológicos. Ela é responsável pelo registro de dados de forma distribuída que propõe a descentralização como parâmetro de segurança e transparência. Assim, tecnologia que no começo animou apenas um pequeno grupo composto por matemáticos, tecnólogos e criptólogos, chegou às massas e expandiu em âmbito e número as possibilidades. 

% 3. Citar o paralelo de aplicabilidade do mundo financeiro atual com as criptomoedas possibilitado pela disponibilidade de APIs das exchanges, falar de traders e grupos de troca de ideias

Visto a sua natureza tecnológica e uma grande variedade de possibilidades no desenvolvimento de ferramentas utilizando Blockchain e criptomoedas, a prática da criação de serviços voltadas ao emprego desta moeda já se tornou comum. A título de exemplo temos a Original My que realiza autenticação de documentos; a Arcade City que é uma alternativa aos aplicativos de carona existentes; a OpenBazaar que é um \textit{marketplace} online sem taxas extras; dentre outros. 

Devido ao sucesso alcançado e ao aumento do poder de compra, surgiu assim, a necessidade da criação de uma bolsa de valores destinada a permutas exclusivas para as criptomoedas, as chamadas Exchanges. De acordo com o site Y, que lista todas as exchanges existentes, valores movimentados e criptomoedas listadas em cada uma, existe atualmente mais de N exchanges, como as brasileiras FoxBit, WallTime, BitcoinTrade, Mercado Bitcoin que movimentam diariamente em média Z Bitcoins sem a necessidade de um intermediário. Levantamento feito por [artigo da globo nos comentários], já são mais que o dobro de investidores neste mercado em relação ao total de CPFs cadastrados na Bolsa de valores Brasileira. 

Pessoas com o propósito de permutar tais moedas com outros usuários com o desejo de obter lucro, principalmente os que operam com análise de gráficos, necessitam acompanhar frequentemente sites, aplicativos e grupos de notícias em aplicativos de mensagens para troca de ideias, informações e análises. 

Existem também aqueles que compartilham análises do mercado, instruções de compra e venda de moedas, de forma paga ou gratuita, se fazendo necessário os negociantes (os chamados traders) realizarem tais operações manualmente, o que pode eventualmente prejudicar o lucro devido a volatilidade destes ativos.

Diante do cenário de crescimento e popularidade do mercado de criptomoedas no Brasil e no mundo, e da possibilidade de utilizar chatbots em aplicativos de mensagem para interação simples e direta com os usuários, uma ferramenta para automação de ordens de compra e venda de criptomoedas se faz oportuna.

% -- Desculpa amor, tirei essa parte pq não consegui encaixar nada nele, nem falar de chatbots de forma detalhada.

% Todo esse processo, em conjunto com a curva de aprendizado de descobrir tudo o que é necessário para se realizar uma transação de criptomoedas, torna o procedimento lento e apresenta riscos devidos aos problemas de informações de entrada, possíveis fraudes e roubos. 
% Então, com a proposta de tornar todo o processo simples e rápidas surgiu, chats robotizados que permitem a automação de funções que anteriormente apenas um ser humano podia fazer.

\section{Objetivos}
\subsection{Objetivo Geral}
O objetivo geral deste trabalho é prover uma aplicação Android.

\subsection{Objetivos Específicos}
\begin{lista}
  \item Apresentar referencial teórico sobre as tecnologias a serem utilizadas, bem como o mercado de criptomoedas;
  \item Planejar e desenvolver um chatbot com as funcionalidades necessárias para gerência de clientes e operações de compra e venda de criptomoedas;
  \item Disponibilizar a ferramenta proposta para a plataforma de comunicação instantânea Telegram.
\end{lista}