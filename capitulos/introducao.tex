% ----------------------------------------------------------
% Introdução
% ----------------------------------------------------------
\chapter{Introdução}

Nesse capítulo será feita uma contextualização sobre as criptomoedas, de onde ele surgiu, evoluiu e como este mercado se apresenta. Em seguida é apresentado uma justificativa de como é possível agregar valores como segurança, transparência, velocidade e simplicidade a pessoas que se utilizam deste mercado para compra e venda destes novos ativos digitais.

\section{Contextualização}
% Falar sobre CriptoMoedas, ChatBots, Traders, Exchanges
% Maybe contextualizar As revoluções como internet, netflix, bitcoin para o dinheiro,
% Falar da atual rotina de quem acompanha tais grupos, ou é cliente de serviços que guardam seu dinheiro, divulgar serviços que sumiram ou dinheiro de clientes, numero de pessoas que utilizam de criptomoedas 

% ======================================================
% # Tópicos a abordar:
% 1. Breve Explicação da evolução da moeda e da internet
% 2. Criação, benefícios e utilização de criptomoedas

A humanidade utilizou diversos instrumentos tangíveis como meio de troca durante a sua evolução, de conchas e sal até pedaços de papel com os quais somos familiares. Contudo, no decorrer da história, o desenvolvimento tecnológico eliminou, em um processo de ordem espontânea, grande parte do meio físico de troca, tornando-o eletrônica e preservando apenas a parte mais importante: a informação e o valor atribuído a este instrumento pela sociedade. 

O conceito destes instrumentos de troca, as moedas, é determinado pela Constituição Brasileira no artigo 21 do inciso VII e a Lei. 9.069, de 29 de junho de 1995, prediz as condições para a sua emissão demostrando que a moeda é instituída por imposição legal. Em 2013, com o surgimento e ascensão das moedas eletrônicas, a lei nº 12.865, de 9 de outubro de 2013, tratou diretamente de disciplina-las conceituando-as como “recursos armazenados em dispositivo ou sistema eletrônico que permitem aos usuário final efetuar transação de pagamento”.

Seguindo com o avanço das tecnologias virtuais e da programação, na década passada um novo conceito de moeda surgiu, diferente das moedas eletrônicas, com o intuito de revolucionar as trocas monetárias de todo o mundo sem a intervenção de governos ou de instituições particulares com o propósito de lucrar com esse movimento. Em 31 de Outubro de 2008, é publicado um \textit{paper} intitulado "Bitcoin: Um Sistema de Dinheiro Eletrônico Ponto-a-Ponto"  ("Bitcoin: A Peer-to-Peer Electronic Cash System" em inglês), escrito por um autor sob o pseudônimo de Satoshi Nakamoto, dando inicio as criptomoedas.

Em pouco tempo, a confiança no sistema divulgado começa a crescer e a primeira compra com Bitcoin é feita: Duas pizzas são compradas por 10.000 Bitcoin, inaugurando o Bitcoin como um meio de pagamento efetivo. Desde então muita coisa mudou, o Bitcoin passou a ser aceito mundialmente por empresas, comércios e prestadores de serviços e muitas outras criptomoedas foram criadas a partir do Bitcoin com o propósito de trazer melhorias. 

A tecnologia por trás de seu funcionamento, a Blockchain, está revolucionando inúmeros setores além do financeiro, sendo estudado por gigantes tecnológicos. Ela é responsável pelo o registro de dados de forma distribuída que propõe a descentralização como parâmetro de segurança. Assim, tecnologia que no começo animou apenas um pequeno grupo composto por matemáticos, tecnólogos e criptólogos, chegou às massas e expandiu em âmbito e número as possibilidades. 

% 3. Citar o paralelo de aplicabilidade do mundo financeiro atual com as criptomoedas possibilitado pela disponibilidade de APIs das exchanges, falar de traders e grupos de troca de ideias

Visto a sua natureza tecnológica e uma grande variedade de possibilidades no desenvolvimento de ferramentas utilizando Blockchain e criptomoedas, a prática da criação de serviços voltadas o emprego desta moeda já se tornou comum. A título de exemplo temos a Original My que realiza autenticação de documentos; a Arcade City que é uma alternativa aos aplicativos de carona existentes; a OpenBazaar que é um \textit{marketplace} online sem taxas, restrições ou terceiro; dentre outros. 

Devido ao sucesso alcançado e ao aumento do poder de compra, surgiu assim, a necessidade da criação de uma bolsa de valores destinada a permuta exclusivas para as criptomoedas, as chamadas Exchanges. De acordo com o site Y, que lista todas as exchanges existentes, valores movimentados e criptomoedas listadas em cada uma, existe atualmente mais de N exchanges, como as brasileiras FoxBit, WallTime, BitcoinTrade, Mercado Bitcoin que movimentam diariamente em média Z Bitcoins sem a necessidade de um intermediário. Levantamento feito por Criaturas, já são mais que o dobro de investidores neste mercado em relação ao total de CPFs cadastrados na Bolsa de valores Brasileira. 

Pessoas com o propósito de permutar tais moedas com outros usuários com o desejo de obter lucro, principalmente os que operam com análise de gráficos, necessitam acompanhar frequentemente sites, aplicativos e grupos em comunicadores instantâneos para troca de ideias, informações e análises. Todo esse processo, em conjunto com a curva de aprendizado de descobrir tudo o que é necessário para se realizar uma transação de criptomoedas, torna o procedimento lento e apresenta riscos devidos aos problemas de informações de entrada, possíveis fraudes e roubos. 

Então, com a proposta de tornar todo o processo simples e rápidas surgiu, com o advento da Inteligência Artificial (IA), chats robotizados que permitem a automação de funções que anteriormente apenas um ser humano podia fazer.



% paragrafo para ligar o assunto com chatbot 
% falar do problema a ser resolvido aqui
% falar exclusivamente de chatbots
% 4. Citar o uso de comunicadores instantâneos para esta troca de ideias e informações comuns em corretoras, devido ao crescente uso de celulares, internet, apps, etc.


% 5. Contextualizar a forma de utilização e linkar com a proposta do trabalho que será justificada no tópico seguinte
% ======================================================

% --- TCC DO RAFAO
% Em sua essência, Financiamento Coletivo é parte de um conceito mais amplo chamado Contribuição Colaborativa (ou Colaboração Coletiva - do inglês \emph{Crowdsourcing}). Em plataformas dessa modalidade de colaboração utiliza-se do "coletivo" para se obter ideias, \emph{feedback} e soluções para problemas através de uma chamada ampla via Internet e a custo zero, ou bastante reduzidos.

% Sites como \emph{The Mechanical Turk}\footnote{https://www.mturk.com} oferecem uma plataforma onde pessoas ou organizações podem colocar pedidos de micro-trabalhos, como votar na qualidade de traduções ou classificar vídeos em relação a seu conteúdo. Como recompensa a pessoa que faz essas tarefas recebe micro-pagamentos de, por exemplo, U\$ 0,15 (quinze centavos de dólar) por vídeo classificado. Outros como o \emph{Kickstarter}\footnote{https://www.kickstarter.com} oferecem uma plataforma e rede social para arrecadar fundos para projetos normalmente relacionados a artes audiovisuais e atualmente é uma das mais prominentes plataformas de \emph{crowdfunding}. Em Novembro de 2016, nove das dez mais bem financiadas campanhas de \emph{crowdfunding} foram feitas via Kickstarter\footnote{Consultado em 31 jan. 2017 em http://crowdfundingblog.com/most-successful-crowdfunding-projects/}. Juntas essas campanhas arrecadaram mais de U\$ 102 milhões.

% Além de casos consolidados como esses, grandes empresas da Internet como Google, Amazon e Facebook estão apostando cada vez mais em soluções de \emph{Crowdsourcing} para inúmeras tarefas que necessitam de interação humana e para treinamento de Inteligências Artificiais. Um caso recente é o aplicativo Android Google \emph{Crowdsourcing} \cite{cnet-google-crowdsourcing} que dá aos usuários pequenas tarefas para auxiliar o aprendizado das inteligências artificiais por trás dos produtos Google. Tarefas como reconhecimento de escrita, interpretação de textos em fotografias, avaliação de traduções e várias outras estão disponíveis para serem resolvidas através do aplicativo e treinarem as várias inteligências artificiais por trás dos produtos da empresa.

% Dessa nova forma de colaboração surgiu, com o passar do tempo, um novo fenômeno: o Financiamento Colaborativo, ou \emph{Crowdfunding}. Ambos utilizam o poder de várias pessoas engajadas e pequenas contribuições de um grande número de pessoas para atingirem seus objetivos \cite{crowdfunding-culture}. Em contraste ao \emph{Crowdsourcing}, onde as pessoas financiam a si mesmas ou são financiadas com recompensas para fazerem o trabalho, o fluxo de capital no \emph{Crowdfunding} é o contrário. Projetos que utilizam \emph{crowdfunding} pedem, através de plataformas online, pequenas contribuições financeiras de contribuidores individuais ou mesmo de investidores. O objetivo desses projetos normalmente é fazer algo mais pessoal como produção artística ou apoiar a produção de softwares como, por exemplo, jogos de forma mais independente. Assim é criado um novo modelo de investimento que circunver formas tradicionais de investimento como empréstimos junto a bancos, \emph{venture capital} (capital de risco) e afins \cite{belleflamme2010}.

% A primeira plataforma de \emph{crowdfunding} a ter sucesso e ser responsável por iniciar de fato o mercado de \emph{crowdfunding} nos Estados Unidos foi o \emph{ArtistShare}\footnote{http://www.artistshare.com/} em 2003 \cite{freedman2015brief}. De autoria de Brian Camelio, um músico e programador de Boston, o \emph{ArtistShare} foi um \emph{website} onde músicos podiam buscar doações de seus fãs para possibilitar aos artistas a gravação e produção digital de música. Eventualmente o site se tornou uma plataforma de financiamento coletivo para artistas audiovisuais, fotógrafos e músicos.

% Seguindo essa tendencia, em 2005, Matt e Jessica Flannery idealizaram e lançaram o que é considerada a primeira plataforma para \emph{crowdfunding} social: Kiva\footnote{http://www.kiva.org}, uma agência de financiamento que utiliza \emph{crowdfunding} para prover micro crédito a empreendedores pobres em países em desenvolvimento, mais notavelmente no Leste da África, India e Ásia Central. O Kiva é uma empresa sem fins lucrativos (\emph{non-profit}) e plataforma tecnológica que liga pessoas que mesmo com pouco dinheiro, possuem a vontade de ajudar a pessoas que necessitam de recursos para ter um mínimo de qualidade de vida ou oportunidade \cite{flannery2007kiva}. A ideia começou quando os criadores do Kiva, durante uma viagem a África, conheceram o dono de uma peixaria na Etiópia que não tinha como melhorar seus lucros, pois não tinha dinheiro para comprar uma passagem de ônibus e dependia de um atravessador para comprar peixes.

% No Brasil o mercado de \emph{Crowdfunding} está ainda em seu começo, mas cresce a cada ano. Como exposto por \citeauthor{globo-financiamento}, depois de passar mais de cinco anos procurando sem sucesso por investidores dispostos a financiar sua ideia, o arquiteto Márcio Cerqueira resolveu apostar em financiamento coletivo. A decisão foi fundamental para tirar do papel o Mola, espécie de \emph{Lego} que ajuda estudantes de arquitetura a entender melhor as estruturas de edifícios. O objetivo inicial era de levantar R\$ 50 mil, mas o projeto teve mais de 1.500 apoiadores e acabou arrecadando R\$ 600 mil, mais de 10 vezes a meta inicial, algo bastante incomum quando se busca financiamento com grandes investidores. Já no projeto Catarse\footnote{https://www.catarse.me}, uma das maiores plataformas nacionais de \emph{crowdfunding}, o volume arrecadado em 2016 foi de R\$ 16.2 milhões, um crescimento de 41\% em relação a 2015 \cite{catarse-retrospectiva2016}, e 134.827 pessoas apoiaram projetos na plataforma e desses 77.98\% apoiaram pela primeira vez (105.150 pessoas).

% Diante do cenário de crescimento da modalidade de \emph{crowdfunding} no Brasil e da possibilidade de se utilizar dessa modalidade para financiamento de projetos sociais, uma ferramenta para fazer isso se faz tanto oportuna como necessária.



\section{Justificativa}

% Colocar gráficos referentes ao crescendo uso de criptomoedas, número de negociações, comparativos entre telegram e whats app, falar da segurança de não ter permissão de saque, das vantagens da automação e da atual rotina de um trader e um cliente, Podemos citar a transformação do mercado de ações, citar grupo de trade de corretoras brasileiras

% --- TCC DO RAFAO
% A situação atual das ONGs\footnote{Organizações Não-Governamentais}, como normalmente são chamadas organizações sem fins lucrativos, inclui dificuldades de várias ordens, mas as mais comuns, e que muitas vezes impedem a iniciativa de continuar ou até mesmo começar são dificuldades em identificar fontes de financiamento e captar recursos. Elas enfrentam críticas sobre o papel que ocupam na economia e na sociedade, sua relação com o governo e as empresas \cite{GOUVEIA2007}. Além desses problemas, muitas vezes ONGs têm problemas para captação de recursos junto a pessoas físicas, pois normalmente seus gestores acreditam que o voluntariado é o bastante, como indicado por \citeauthor{modeloGestaoONG}. No Brasil, atualmente se vê um crescimento do trabalho voluntário a um ponto que alguns autores \cite{fagundes2012repercussoes} consideram isso como influência negativa a implementação de determinadas politicas sociais governamentais para diminuição da pobreza. Entende-se que se as próprias pessoas estão se mobilizando para resolver alguns problemas sociais, os mesmos se tornam menores e consequentemente menos recursos necessitam ser alocados para isso.

% Apesar do crescimento das plataformas nacionais de \emph{crowdfunding}, da quantidade de projetos e de apoiadores, essas plataformas têm taxas de uso comumente superiores a 10\%. Parte deste percentual são custos do \emph{gateway} de pagamentos, empresas que prestam serviço de recebimento de pagamentos online, utilizado pelo serviço como, por exemplo, o MoIP\footnote{3,49\% + R\$0,69 a 5,49\% + R\$0,69 por transação - Consultado em 27/01/2017 em https://moip.com.br/tarifas/}. Além desses custos, as plataformas nacionais não disponibilizam opção de escolha em relação a qual \emph{gateway} de pagamento o projeto deseja utilizar. O Catarse, por exemplo, utiliza o \emph{gateway} Pagar.me\footnote{Consultado em 27/01/2017 em http://crowdfunding.catarse.me/nossa-taxa}. Outro grande serviço nacional, o Kickante utiliza MoIP como \emph{gateway} de pagamento\footnote{Consultado em 27/01/2017 em https://www.kickante.com.br/termos/termos-de-uso, item 9.1.2}. Em virtude dessa falta de escolha, o projeto deste trabalho traz a inovação da escolha do \emph{gateway} de pagamento utilizado a fim de permitir o uso do \emph{gateway} com a melhor relação custo/benefício à instituição a qual pode ter, por exemplo, parceria com o mesmo.

% Ante os custos apresentados, as dificuldades envolvidas em outras formas de financiamento e as facilidades e potenciais benefícios, este trabalho se propõe a criar uma plataforma de \emph{crowdfunding} de código aberto chamada Ajuda.Ai que acarrete o mínimo custo possível para os projetos financiados pela plataforma, focada para organizações de pequeno porte. Para esse objetivo, um dos pontos principais e diferenciais da ferramenta é a possibilidade da escolha de qual \emph{gateway} de pagamento será usado para processar as doações. Além disso, nenhum custo fora os embutidos pelo próprio \emph{gateway} de pagamento será acrescido às doações, dando assim uma maior margem ao projeto sobre as doações recebidas.

% Custos de manutenção e hospedagem do projeto serão cobertos inicialmente através de capital pessoal e, com o crescimento do Ajuda.Ai, há a possibilidade de se utilizar a própria plataforma para captação de recursos para mantê-lo ou, igualmente ao projeto Kiva, buscar investimento junto a investidores anjo ou filantropos.



\section{Objetivos}
\subsection{Objetivo Geral}
O objetivo geral deste trabalho é prover uma ferramenta de chatbot destinada a \emph{Traders} que divulgam oportunidades de compra e venda de criptomoeda. A ferramenta CallTraderBot irá prover replicação, acompanhamento e automação destas operações, buscando assim trazer  eficiência e valor aos usuários que acompanham tais oportunidades.

\subsection{Objetivos Específicos}
\begin{lista}
  \item Apresentar referencial teórico sobre as tecnologias a serem utilizadas, bem como o mercado de criptomoedas;
  \item Planejar e desenvolver um chatbot com as funcionalidades necessárias para gerência de clientes e operações de compra e venda de criptomoedas;
  \item Disponibilizar a ferramenta proposta para a plataforma de comunicação instantânea Telegram.
\end{lista}

\section*{Resumo}
Neste capítulo foi apresentada uma contextualização sobre o problema tratado neste trabalho e a justificativa de tal assunto, que pode ser resumida como sendo necessidade de uma alternativa moderna e simplificada para automação e replicação de compra e venda e criptomoedas. Ao final, foram detalhados os objetivos gerais e específicos do trabalho.

Os próximos capítulos estão organizados da seguinte forma:

\begin{lista}
  \item \textbf{Fundamentação Teórica:} Neste capítulo são apresentados todos os conceitos teóricos utilizados no desenvolvimento da solução proposta no presente trabalho;
  \item \textbf{Metodologia:} Capítulo dedicado a apresentação da metodologia de desenvolvimento utilizada, tecnologias, e demais tecnologias utilizadas no processo de construção;
  \item \textbf{CallTraderBot:} Capítulo dedicado a apresentação da solução implementada, detalhando funcionalidades, utilização e demais informações a respeito da ferramenta;
  \item \textbf{Conclusão:} Neste capítulo é feita a conclusão do trabalho dado seus objetivos propostos e são listados os trabalhos futuros para melhorar e/ou expandir a utilização da solução.
\end{lista}