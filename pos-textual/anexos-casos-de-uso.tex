%%%%%%%%%%%%%%%%%%%%%%%%%%%%%%%%%%%%%%%%%%%%%%%
%% Anexo B (como o A foi retirado, esse que realmente será o A)
%%%%%%%%%%%%%%%%%%%%%%%%%%%%%%%%%%%%%%%%%%%%%%%
\chapter{Especificação de Casos de Uso} \label{anexo:b}

%%%%%%%%%%%%%%%%%%%%%%%%%%%%%%%%%%%%%%%%%%%%%%%
%% Anexo B - Caso de Uso 01
%% Aqui são usado os asteriscos para retirar essas seções/subseções do Sumário
%%%%%%%%%%%%%%%%%%%%%%%%%%%%%%%%%%%%%%%%%%%%%%%
\section*{Especificação do Caso de Uso 01 --- Autenticar}
\subsection*{Objetivo}
Este documento tem por objetivo descrever todos os fluxos envolvidos no Caso de Uso 01 - Autenticar. São listados e detalhados todos os atores, fluxos, requisitos funcionais e não-funcionais.

\subsection*{Identificação dos Atores}
Esta seção lista e descreve todos os atores envolvidos nos fluxos que compõem o Caso de Uso 01 - Autenticar.
\begin{lista}
  \item \textbf{Usuário}: Qualquer pessoa que utiliza o aplicativo;
\end{lista}

\subsection*{Identificação dos Fluxos}
Esta seção lista e descreve todos os fluxos que compõem o caso de uso.
\begin{lista}
  \item \textbf{Autenticar com Google}: Este fluxo descreve como se dá a autenticação do usuário usando sua conta Google.
  \item \textbf{Acessar como anônimo}: Este fluxo descreve como se dá a autenticação do usuário que deseja acessar sem identificação.
\end{lista}

\subsection*{Detalhamento dos Fluxos}
\begin{lista}
  \item \textbf{Autenticar com Google}: Este fluxo especifica a ação de iniciar uma sessão no aplicativo. O usuário fornece suas credenciais de sua conta Google para ter um acesso identificado no aplicativo.
    \begin{itemize}
    \item \textbf{Atores}: Usuário;
    \item \textbf{Pré-Condições}: Nenhuma;
    \item \textbf{Pós-Condições}: Nenhuma;
    \item \textbf{Requisitos Funcionais}: O sistema deve prover uma interface que faz chamada ao serviço da Google responsável por autenticação.
    \end{itemize}
	
    \textbf{Fluxo Básico}
    \begin{enumerate}
    \item O aplicativo exibe uma tela com a opção de autenticação utilizando suas credenciais Google ou modo anônimo;
    \item O Usuário decide acessar o sistema utilizando sua conta Google;
    \item O aplicativo executa e exibe em primeiro plano o serviço de autenticação da Google, que por sua vez possui seu próprio fluxo de autenticação;
    \item O serviço de autenticação da Google retorna o status da solicitação juntamente com alguns dados básicos do usuário;
    \item O aplicativo valida os dados;
    \item O aplicativo registra todas as informações no Banco de Dados;
    \item O Usuário é direcionado a tela principal do aplicativo;
    \item O caso de uso se encerra.
    \end{enumerate}
    
    \textbf{Fluxo Alternativo A} \\
    No Passo 5, caso exista algum erro de comunicação entre o aplicativo e o serviço de autenticação da Google:
    \begin{enumerate}
    \item Uma mensagem de erro é exibida ao Usuário informando a falha;
    \item O fluxo retorna ao passo 1.
    \end{enumerate}
    
    \textbf{Fluxo Alternativo B} \\
    No Passo 5, caso exista algum erro de autenticação:
    \begin{enumerate}
    \item Uma mensagem de erro é exibida ao Usuário informando da falha;
    \item O fluxo retorna ao passo 1.
    \end{enumerate}
\end{lista}
\pagebreak

%%%%%%%%%%%%%%%%%%%%%%%%%%%%%%%%%%%%%%%%%%%%%%%
%% Anexo B - Caso de Uso 02
%%%%%%%%%%%%%%%%%%%%%%%%%%%%%%%%%%%%%%%%%%%%%%%
\section*{Especificação do Caso de Uso 02 --- Participar de torneios}
\subsection*{Objetivo}
Este documento tem por objetivo descrever todos os fluxos envolvidos no Caso de Uso 02 - Participar de torneios. São listados e detalhados todos os atores, fluxos, requisitos funcionais e não-funcionais.

\subsection*{Identificação dos Atores}
Esta seção lista e descreve todos os atores envolvidos nos fluxos que compõem o Caso de Uso 02 - Participar de torneios.
\begin{lista}
  \item \textbf{Usuário}: Qualquer pessoa autenticada que utiliza o aplicativo;
\end{lista}

\subsection*{Identificação dos Fluxos}
Esta seção lista e descreve todos os fluxos que compõem o caso de uso.
\begin{lista}
  \item \textbf{Participar de Torneios}: Este fluxo descreve como se dá a participação de um usuário em um Torneio.
\end{lista}

\subsection*{Detalhamento dos Fluxos}
\begin{lista}
  \item \textbf{Participar de Torneios}: .
    \begin{itemize}
    \item \textbf{Atores}: Usuário;
    \item \textbf{Pré-Condições}: O usuário estar autenticado junto ao aplicativo;
    \item \textbf{Pós-Condições}: Usuário inscrito num torneio;
    \item \textbf{Requisitos Funcionais}: O aplicativo deve prover uma interface inscrição em torneios.
    \end{itemize}
	
    \textbf{Fluxo Básico}
    \begin{enumerate}
    \item O usuário decide participar de um dos torneios disponíveis na página principal;
    \item O usuário acessa o detalhe do torneio;
    \item O usuário clica em participar;
    \item O aplicativo envia os dados ao Sistema;
    \item O Sistema registra a inscrição no SGBD;
    \item O Sistema retorna ao Aplicativo o status da solicitação;
    \item Uma mensagem é exibida ao Usuário informando o resultado da inscrição;
    \item O caso de uso se encerra.
    \end{enumerate}
    
    \textbf{Fluxo Alternativo A} \\
    Em qualquer passo, caso exista algum erro de comunicação entre o aplicativo e servidor:
    \begin{enumerate}
    \item Uma mensagem de erro é exibida ao Usuário informando da falha.
    \end{enumerate}
\end{lista}
\pagebreak

%%%%%%%%%%%%%%%%%%%%%%%%%%%%%%%%%%%%%%%%%%%%%%%
%% Anexo B - Caso de Uso 03
%%%%%%%%%%%%%%%%%%%%%%%%%%%%%%%%%%%%%%%%%%%%%%%
\section*{Especificação do Caso de Uso 03 --- Visualizar notificações}

\subsection*{Objetivo}
Este documento tem por objetivo descrever todos os fluxos envolvidos no Caso de Uso 03 - Visualizar notificações. São listados e detalhados todos os atores, fluxos, requisitos funcionais e não-funcionais.

\subsection*{Identificação dos Atores}
Esta seção lista e descreve todos os atores envolvidos nos fluxos que compõem o Caso de Uso 03 - Visualizar notificações.
\begin{lista}
  \item \textbf{Usuário}: Qualquer pessoa autenticada que utiliza o sistema.
\end{lista}

\subsection*{Identificação dos Fluxos}
Esta seção lista e descreve todos os fluxos que compõem o caso de uso.
\begin{lista}
  \item \textbf{Listagem de Notificações}: Este fluxo descreve como se dá o acesso a listagem de notificações recebidas pelo Aplicativo;
\end{lista}

\subsection*{Detalhamento dos Fluxos}
\begin{lista}
  \item \textbf{Listagem de Notificações}: Este fluxo especifica a ação de listar as notificações recebidas pelo aplicativo.
    \begin{itemize}
    \item \textbf{Atores}: Usuário;
    \item \textbf{Pré-Condições}: O usuário estar autenticada junto ao Aplicativo;
    \item \textbf{Pós-Condições}: Nenhuma;
    \item \textbf{Requisitos Funcionais}: O Aplicativo deve prover uma interface para listagem das Notificações e Alertas feitas ao usuário autenticado.
    \end{itemize}
	
    \textbf{Fluxo Básico}
    \begin{enumerate}
    \item O usuário clica no ícone de notificações na tela principal;
    \item O Aplicativo exibe todas as notificações disponíveis ao usuário;
    \item O usuário clica em uma das notificações;
    \item É exibida os detalhes da notificação ao usuário;
    \item O caso de uso se encerra.
    \end{enumerate}
\end{lista}

%%%%%%%%%%%%%%%%%%%%%%%%%%%%%%%%%%%%%%%%%%%%%%%%%%%%%%%%%%%%%%%

%%%%%%%%%%%%%%%%%%%%%%%%%%%%%%%%%%%%%%%%%%%%%%%
%% Anexo B - Caso de Uso 04
%%%%%%%%%%%%%%%%%%%%%%%%%%%%%%%%%%%%%%%%%%%%%%%
\section*{Especificação do Caso de Uso 04 --- Comprar e vender ativos}
\subsection*{Objetivo}
Este documento tem por objetivo descrever todos os fluxos envolvidos no Caso de Uso 04 - Comprar e vender ativos. São listados e detalhados todos os atores, fluxos, requisitos funcionais e não-funcionais.

\subsection*{Identificação dos Atores}
Esta seção lista e descreve todos os atores envolvidos nos fluxos que compõem o Caso de Uso 03 - Comprar e vender ativos.
\begin{lista}
  \item \textbf{Usuário}: Qualquer usuário autenticado que utilize o aplicativo;
\end{lista}

\subsection*{Identificação dos Fluxos}
Esta seção lista e descreve todos os fluxos que compõem o caso de uso.
\begin{lista}
  \item \textbf{Comprar ativos financeiros}: Este fluxo descreve como se dá a compra de ativos financeiros pelo usuário;
  \item \textbf{Vender ativos financeiros}: Este fluxo descreve como se dá a venda de ativos financeiros pelo usuário.
\end{lista}

\subsection*{Detalhamento dos Fluxos}
\begin{lista}
  \item \textbf{Comprar ativos financeiros}: Este fluxo especifica a ação de listar os ativos financeiros e efetuar uma ordem de compra de um determinado item informando a quantidade e o valor do mesmo;
    \begin{itemize}
    \item \textbf{Atores}: Usuário;
    \item \textbf{Pré-Condições}: O usuário estar autenticado; O usuário possuir saldo;
    \item \textbf{Pós-Condições}: Nenhuma;
    \item \textbf{Requisitos Funcionais}: O Sistema deve prover uma interface para listagem de ativos financeiros e a opção de compra do mesmo.
    \end{itemize}
	
    \textbf{Fluxo Básico}
    \begin{enumerate}
    \item O usuário acessa a tela referente as negociações de ativos financeiros;
    \item O usuário seleciona um ativo financeiro;
    \item O usuário clica na opção de compra;
    \item O usuário informa os dados solicitados: quantidade e valor;
    \item O Aplicativo envia os dados ao Sistema;
    \item O Sistema valida os dados;
    \item O Sistema retorna ao Aplicativo com um status da ação requisitada;
    \item O Aplicativo exibe uma mensagem de sucesso ao usuário;
    \item O caso de uso se encerra.
    \end{enumerate}
    
    \textbf{Fluxo Alternativo A} \\
    Em qualquer passo, caso exista algum erro de comunicação entre Aplicativo e Sistema:
    \begin{enumerate}
    \item Uma mensagem de erro é exibida ao Usuário informando da falha.
    \end{enumerate}
    
    \textbf{Fluxo Alternativo B} \\
    No Passo 6, caso exista algum erro de validação das entradas ou falta de saldo disponível:
    \begin{enumerate}
    \item Todos os erros de validação encontrados são retornados ao Cliente;
    \item O fluxo retorna ao passo 4.
    \end{enumerate}
  
  
\end{lista}

%%%%%%%%%%%%%%%%%%%%%%%%%%%%%%%%%%%%%%%%%%%%%%%%%%%%%%%%%%%%%%%

% %%%%%%%%%%%%%%%%%%%%%%%%%%%%%%%%%%%%%%%%%%%%%%%
% %% Anexo B - Caso de Uso 05
% %%%%%%%%%%%%%%%%%%%%%%%%%%%%%%%%%%%%%%%%%%%%%%%
% \section*{Especificação do Caso de Uso 05 --- Visualizar histórico} [PENDENTE]
% --- O usuário pode visualizar o histórico de suas negociações e evolução de seu saldo;
% \subsection*{Objetivo}
% Este documento tem por objetivo descrever todos os fluxos envolvidos no Caso de Uso 03 - Visualizar histórico. São listados e detalhados todos os atores, fluxos, requisitos funcionais e não-funcionais.

% \subsection*{Identificação dos Atores}
% Esta seção lista e descreve todos os atores envolvidos nos fluxos que compõem o Caso de Uso 03 - Visualizar histórico.
% \begin{lista}
%   \item \textbf{Instituição}: Usuário cadastrado no sistema e que possua pelo menos 01 (uma) Instituição vinculada a ele;
%   \item \textbf{Gateway}: \emph{Gateway} de Pagamento selecionado pela Instituição;
%   \item \textbf{Cliente}: Interface através da qual o Usuário utiliza o Ajuda.Ai.
% \end{lista}

% \subsection*{Identificação dos Fluxos}
% Esta seção lista e descreve todos os fluxos que compõem o caso de uso.
% \begin{lista}
%   \item \textbf{Listagem Mensal de Doações}: Este fluxo descreve como se dá a criação de uma listagem mensal de Doações recebidas pela Instituição;
%   \item \textbf{Dados de uma Doação}: Este fluxo descreve como se dá a exibição de dados sobre uma Doação em Particular.
% \end{lista}

% \subsection*{Detalhamento dos Fluxos}
% \begin{lista}
%   \item \textbf{Listagem Mensal de Doações}: Este fluxo especifica a ação de listar as doações recebidas por uma Instituição em um determinado mês de um determinado ano. A Instituição fornece dados ao cliente o qual, junto ao sistema, cria uma lista, geralmente tabular, das doações recebidas.
%     \begin{itemize}
%     \item \textbf{Atores}: Instituição, Cliente;
%     \item \textbf{Pré-Condições}: Instituição cadastrada no Sistema, Instituição estar autenticada junto ao Sistema;
%     \item \textbf{Pós-Condições}: Nenhuma;
%     \item \textbf{Requisitos Funcionais}: O Sistema deve prover uma interface para listagem das Doações feitas a uma Instituição.
%     \end{itemize}
	
%     \textbf{Fluxo Básico}
%     \begin{enumerate}
%     \item A Instituição necessita de uma listagem das doações de um mês;
%     \item A Instituição navega a uma página para listagem de doações;
%     \item O Cliente solicita os seguintes dados: De qual Instituição deve ser a lista, Mês e Ano da listagem e um dos itens de uma lista de Estados das Doações: Todos, Pago, Pronto para Receber e Cancelado;
%     \item A Instituição informa os dados solicitados;
%     \item O Cliente envia os dados ao Sistema;
%     \item O Sistema valida os dados;
%     \item O Sistema retorna ao Cliente os dados da listagem requisitada;
%     \item O caso de uso se encerra.
%     \end{enumerate}
    
%     \textbf{Fluxo Alternativo A} \\
%     Em qualquer passo, caso exista algum erro de comunicação entre Cliente e Sistema:
%     \begin{enumerate}
%     \item Uma mensagem de erro é exibida ao Usuário informando da falha.
%     \end{enumerate}
    
%     \textbf{Fluxo Alternativo B} \\
%     No Passo 6, caso exista algum erro de validação das entradas:
%     \begin{enumerate}
%     \item Todos os erros de validação encontrados são retornados ao Cliente;
%     \item O fluxo retorna ao passo 2.
%     \end{enumerate}
  
  
  
%   \item \textbf{Dados de uma Doação}: Este fluxo especifica a ação de ver dados de uma única doação específica. Após escolher qual doação será exibida, a Instituição fornece os dados necessários ao cliente o qual, junto ao sistema, pega e exibe as informações requeridas.
%     \begin{itemize}
%     \item \textbf{Atores}: Instituição, Cliente;
%     \item \textbf{Pré-Condições}: Instituição cadastrada no Sistema, Instituição estar autenticada junto ao Sistema, Haverem Doações feitas a Instituição;
%     \item \textbf{Pós-Condições}: Nenhuma;
%     \item \textbf{Requisitos Funcionais}: O Sistema deve prover uma interface para expor dados de um pagamento.
%     \end{itemize}
	
%     \textbf{Fluxo Básico}
%     \begin{enumerate}
%     \item A Instituição decide ver dados de uma doação;
%     \item A Instituição seleciona qual doação será exibida;
%     \item A Instituição navega a uma página para exibição das informações da doação;
%     \item O Cliente requisita ao Sistema os dados da doação;
%     \item O Sistema busca os dados da doação especificada;
%     \item O Sistema retorna ao Cliente os dados da doação;
%     \item O caso de uso se encerra.
%     \end{enumerate}
    
%     \textbf{Fluxo Alternativo A} \\
%     Em qualquer passo, caso exista algum erro de comunicação entre Cliente e Sistema:
%     \begin{enumerate}
%     \item Uma mensagem de erro é exibida ao Usuário informando da falha.
%     \end{enumerate}
    
%     \textbf{Fluxo Alternativo B} \\
%     No Passo 5, caso a doação esteja marcada como Anônima (\codigo{anonymous = true}):
%     \begin{enumerate}
%     \item Todas as informações de identificação do doador são retiradas do objeto (\codigo{payeeName} e \codigo{payeeEmail});
%     \item O fluxo continua ao passo 6.
%     \end{enumerate}
% \end{lista}

% %%%%%%%%%%%%%%%%%%%%%%%%%%%%%%%%%%%%%%%%%%%%%%%%%%%%%%%%%%%%%%%

% %%%%%%%%%%%%%%%%%%%%%%%%%%%%%%%%%%%%%%%%%%%%%%%
% %% Anexo B - Caso de Uso 06
% %%%%%%%%%%%%%%%%%%%%%%%%%%%%%%%%%%%%%%%%%%%%%%%
% \section*{Especificação do Caso de Uso 06 --- Visualizar ranking} [PENDENTE]
% --- É disponibilizado um ranking categórico aos usuários;
% \subsection*{Objetivo}
% Este documento tem por objetivo descrever todos os fluxos envolvidos no Caso de Uso 03 - Visualizar ranking. São listados e detalhados todos os atores, fluxos, requisitos funcionais e não-funcionais.

% \subsection*{Identificação dos Atores}
% Esta seção lista e descreve todos os atores envolvidos nos fluxos que compõem o Caso de Uso 03 - Visualizar ranking.
% \begin{lista}
%   \item \textbf{Instituição}: Usuário cadastrado no sistema e que possua pelo menos 01 (uma) Instituição vinculada a ele;
%   \item \textbf{Gateway}: \emph{Gateway} de Pagamento selecionado pela Instituição;
%   \item \textbf{Cliente}: Interface através da qual o Usuário utiliza o Ajuda.Ai.
% \end{lista}

% \subsection*{Identificação dos Fluxos}
% Esta seção lista e descreve todos os fluxos que compõem o caso de uso.
% \begin{lista}
%   \item \textbf{Listagem Mensal de Doações}: Este fluxo descreve como se dá a criação de uma listagem mensal de Doações recebidas pela Instituição;
%   \item \textbf{Dados de uma Doação}: Este fluxo descreve como se dá a exibição de dados sobre uma Doação em Particular.
% \end{lista}

% \subsection*{Detalhamento dos Fluxos}
% \begin{lista}
%   \item \textbf{Listagem Mensal de Doações}: Este fluxo especifica a ação de listar as doações recebidas por uma Instituição em um determinado mês de um determinado ano. A Instituição fornece dados ao cliente o qual, junto ao sistema, cria uma lista, geralmente tabular, das doações recebidas.
%     \begin{itemize}
%     \item \textbf{Atores}: Instituição, Cliente;
%     \item \textbf{Pré-Condições}: Instituição cadastrada no Sistema, Instituição estar autenticada junto ao Sistema;
%     \item \textbf{Pós-Condições}: Nenhuma;
%     \item \textbf{Requisitos Funcionais}: O Sistema deve prover uma interface para listagem das Doações feitas a uma Instituição.
%     \end{itemize}
	
%     \textbf{Fluxo Básico}
%     \begin{enumerate}
%     \item A Instituição necessita de uma listagem das doações de um mês;
%     \item A Instituição navega a uma página para listagem de doações;
%     \item O Cliente solicita os seguintes dados: De qual Instituição deve ser a lista, Mês e Ano da listagem e um dos itens de uma lista de Estados das Doações: Todos, Pago, Pronto para Receber e Cancelado;
%     \item A Instituição informa os dados solicitados;
%     \item O Cliente envia os dados ao Sistema;
%     \item O Sistema valida os dados;
%     \item O Sistema retorna ao Cliente os dados da listagem requisitada;
%     \item O caso de uso se encerra.
%     \end{enumerate}
    
%     \textbf{Fluxo Alternativo A} \\
%     Em qualquer passo, caso exista algum erro de comunicação entre Cliente e Sistema:
%     \begin{enumerate}
%     \item Uma mensagem de erro é exibida ao Usuário informando da falha.
%     \end{enumerate}
    
%     \textbf{Fluxo Alternativo B} \\
%     No Passo 6, caso exista algum erro de validação das entradas:
%     \begin{enumerate}
%     \item Todos os erros de validação encontrados são retornados ao Cliente;
%     \item O fluxo retorna ao passo 2.
%     \end{enumerate}
  
  
  
%   \item \textbf{Dados de uma Doação}: Este fluxo especifica a ação de ver dados de uma única doação específica. Após escolher qual doação será exibida, a Instituição fornece os dados necessários ao cliente o qual, junto ao sistema, pega e exibe as informações requeridas.
%     \begin{itemize}
%     \item \textbf{Atores}: Instituição, Cliente;
%     \item \textbf{Pré-Condições}: Instituição cadastrada no Sistema, Instituição estar autenticada junto ao Sistema, Haverem Doações feitas a Instituição;
%     \item \textbf{Pós-Condições}: Nenhuma;
%     \item \textbf{Requisitos Funcionais}: O Sistema deve prover uma interface para expor dados de um pagamento.
%     \end{itemize}
	
%     \textbf{Fluxo Básico}
%     \begin{enumerate}
%     \item A Instituição decide ver dados de uma doação;
%     \item A Instituição seleciona qual doação será exibida;
%     \item A Instituição navega a uma página para exibição das informações da doação;
%     \item O Cliente requisita ao Sistema os dados da doação;
%     \item O Sistema busca os dados da doação especificada;
%     \item O Sistema retorna ao Cliente os dados da doação;
%     \item O caso de uso se encerra.
%     \end{enumerate}
    
%     \textbf{Fluxo Alternativo A} \\
%     Em qualquer passo, caso exista algum erro de comunicação entre Cliente e Sistema:
%     \begin{enumerate}
%     \item Uma mensagem de erro é exibida ao Usuário informando da falha.
%     \end{enumerate}
    
%     \textbf{Fluxo Alternativo B} \\
%     No Passo 5, caso a doação esteja marcada como Anônima (\codigo{anonymous = true}):
%     \begin{enumerate}
%     \item Todas as informações de identificação do doador são retiradas do objeto (\codigo{payeeName} e \codigo{payeeEmail});
%     \item O fluxo continua ao passo 6.
%     \end{enumerate}
% \end{lista}

% %%%%%%%%%%%%%%%%%%%%%%%%%%%%%%%%%%%%%%%%%%%%%%%%%%%%%%%%%%%%%%%

% %%%%%%%%%%%%%%%%%%%%%%%%%%%%%%%%%%%%%%%%%%%%%%%
% %% Anexo B - Caso de Uso 07
% %%%%%%%%%%%%%%%%%%%%%%%%%%%%%%%%%%%%%%%%%%%%%%%
% \section*{Especificação do Caso de Uso 07 --- Gerenciar conta} [PENDENTE]
% --- O usuário pode alterar dados referentes a sua conta;
% \subsection*{Objetivo}
% Este documento tem por objetivo descrever todos os fluxos envolvidos no Caso de Uso 03 - Gerenciar conta. São listados e detalhados todos os atores, fluxos, requisitos funcionais e não-funcionais.

% \subsection*{Identificação dos Atores}
% Esta seção lista e descreve todos os atores envolvidos nos fluxos que compõem o Caso de Uso 03 - Gerenciar conta.
% \begin{lista}
%   \item \textbf{Instituição}: Usuário cadastrado no sistema e que possua pelo menos 01 (uma) Instituição vinculada a ele;
%   \item \textbf{Gateway}: \emph{Gateway} de Pagamento selecionado pela Instituição;
%   \item \textbf{Cliente}: Interface através da qual o Usuário utiliza o Ajuda.Ai.
% \end{lista}

% \subsection*{Identificação dos Fluxos}
% Esta seção lista e descreve todos os fluxos que compõem o caso de uso.
% \begin{lista}
%   \item \textbf{Listagem Mensal de Doações}: Este fluxo descreve como se dá a criação de uma listagem mensal de Doações recebidas pela Instituição;
%   \item \textbf{Dados de uma Doação}: Este fluxo descreve como se dá a exibição de dados sobre uma Doação em Particular.
% \end{lista}

% \subsection*{Detalhamento dos Fluxos}
% \begin{lista}
%   \item \textbf{Listagem Mensal de Doações}: Este fluxo especifica a ação de listar as doações recebidas por uma Instituição em um determinado mês de um determinado ano. A Instituição fornece dados ao cliente o qual, junto ao sistema, cria uma lista, geralmente tabular, das doações recebidas.
%     \begin{itemize}
%     \item \textbf{Atores}: Instituição, Cliente;
%     \item \textbf{Pré-Condições}: Instituição cadastrada no Sistema, Instituição estar autenticada junto ao Sistema;
%     \item \textbf{Pós-Condições}: Nenhuma;
%     \item \textbf{Requisitos Funcionais}: O Sistema deve prover uma interface para listagem das Doações feitas a uma Instituição.
%     \end{itemize}
	
%     \textbf{Fluxo Básico}
%     \begin{enumerate}
%     \item A Instituição necessita de uma listagem das doações de um mês;
%     \item A Instituição navega a uma página para listagem de doações;
%     \item O Cliente solicita os seguintes dados: De qual Instituição deve ser a lista, Mês e Ano da listagem e um dos itens de uma lista de Estados das Doações: Todos, Pago, Pronto para Receber e Cancelado;
%     \item A Instituição informa os dados solicitados;
%     \item O Cliente envia os dados ao Sistema;
%     \item O Sistema valida os dados;
%     \item O Sistema retorna ao Cliente os dados da listagem requisitada;
%     \item O caso de uso se encerra.
%     \end{enumerate}
    
%     \textbf{Fluxo Alternativo A} \\
%     Em qualquer passo, caso exista algum erro de comunicação entre Cliente e Sistema:
%     \begin{enumerate}
%     \item Uma mensagem de erro é exibida ao Usuário informando da falha.
%     \end{enumerate}
    
%     \textbf{Fluxo Alternativo B} \\
%     No Passo 6, caso exista algum erro de validação das entradas:
%     \begin{enumerate}
%     \item Todos os erros de validação encontrados são retornados ao Cliente;
%     \item O fluxo retorna ao passo 2.
%     \end{enumerate}
  
  
  
%   \item \textbf{Dados de uma Doação}: Este fluxo especifica a ação de ver dados de uma única doação específica. Após escolher qual doação será exibida, a Instituição fornece os dados necessários ao cliente o qual, junto ao sistema, pega e exibe as informações requeridas.
%     \begin{itemize}
%     \item \textbf{Atores}: Instituição, Cliente;
%     \item \textbf{Pré-Condições}: Instituição cadastrada no Sistema, Instituição estar autenticada junto ao Sistema, Haverem Doações feitas a Instituição;
%     \item \textbf{Pós-Condições}: Nenhuma;
%     \item \textbf{Requisitos Funcionais}: O Sistema deve prover uma interface para expor dados de um pagamento.
%     \end{itemize}
	
%     \textbf{Fluxo Básico}
%     \begin{enumerate}
%     \item A Instituição decide ver dados de uma doação;
%     \item A Instituição seleciona qual doação será exibida;
%     \item A Instituição navega a uma página para exibição das informações da doação;
%     \item O Cliente requisita ao Sistema os dados da doação;
%     \item O Sistema busca os dados da doação especificada;
%     \item O Sistema retorna ao Cliente os dados da doação;
%     \item O caso de uso se encerra.
%     \end{enumerate}
    
%     \textbf{Fluxo Alternativo A} \\
%     Em qualquer passo, caso exista algum erro de comunicação entre Cliente e Sistema:
%     \begin{enumerate}
%     \item Uma mensagem de erro é exibida ao Usuário informando da falha.
%     \end{enumerate}
    
%     \textbf{Fluxo Alternativo B} \\
%     No Passo 5, caso a doação esteja marcada como Anônima (\codigo{anonymous = true}):
%     \begin{enumerate}
%     \item Todas as informações de identificação do doador são retiradas do objeto (\codigo{payeeName} e \codigo{payeeEmail});
%     \item O fluxo continua ao passo 6.
%     \end{enumerate}
% \end{lista}

% %%%%%%%%%%%%%%%%%%%%%%%%%%%%%%%%%%%%%%%%%%%%%%%%%%%%%%%%%%%%%%%

% %%%%%%%%%%%%%%%%%%%%%%%%%%%%%%%%%%%%%%%%%%%%%%%
% %% Anexo B - Caso de Uso 08
% %%%%%%%%%%%%%%%%%%%%%%%%%%%%%%%%%%%%%%%%%%%%%%%
% \section*{Especificação do Caso de Uso 08 ---  Realizar missões} [PENDENTE]
% --- O usuário visualiza as missões diárias disponíveis a fim de ganhar recompensas;
% \subsection*{Objetivo}
% Este documento tem por objetivo descrever todos os fluxos envolvidos no Caso de Uso 03 -  Realizar missões. São listados e detalhados todos os atores, fluxos, requisitos funcionais e não-funcionais.

% \subsection*{Identificação dos Atores}
% Esta seção lista e descreve todos os atores envolvidos nos fluxos que compõem o Caso de Uso 03 -  Realizar missões.
% \begin{lista}
%   \item \textbf{Instituição}: Usuário cadastrado no sistema e que possua pelo menos 01 (uma) Instituição vinculada a ele;
%   \item \textbf{Gateway}: \emph{Gateway} de Pagamento selecionado pela Instituição;
%   \item \textbf{Cliente}: Interface através da qual o Usuário utiliza o Ajuda.Ai.
% \end{lista}

% \subsection*{Identificação dos Fluxos}
% Esta seção lista e descreve todos os fluxos que compõem o caso de uso.
% \begin{lista}
%   \item \textbf{Listagem Mensal de Doações}: Este fluxo descreve como se dá a criação de uma listagem mensal de Doações recebidas pela Instituição;
%   \item \textbf{Dados de uma Doação}: Este fluxo descreve como se dá a exibição de dados sobre uma Doação em Particular.
% \end{lista}

% \subsection*{Detalhamento dos Fluxos}
% \begin{lista}
%   \item \textbf{Listagem Mensal de Doações}: Este fluxo especifica a ação de listar as doações recebidas por uma Instituição em um determinado mês de um determinado ano. A Instituição fornece dados ao cliente o qual, junto ao sistema, cria uma lista, geralmente tabular, das doações recebidas.
%     \begin{itemize}
%     \item \textbf{Atores}: Instituição, Cliente;
%     \item \textbf{Pré-Condições}: Instituição cadastrada no Sistema, Instituição estar autenticada junto ao Sistema;
%     \item \textbf{Pós-Condições}: Nenhuma;
%     \item \textbf{Requisitos Funcionais}: O Sistema deve prover uma interface para listagem das Doações feitas a uma Instituição.
%     \end{itemize}
	
%     \textbf{Fluxo Básico}
%     \begin{enumerate}
%     \item A Instituição necessita de uma listagem das doações de um mês;
%     \item A Instituição navega a uma página para listagem de doações;
%     \item O Cliente solicita os seguintes dados: De qual Instituição deve ser a lista, Mês e Ano da listagem e um dos itens de uma lista de Estados das Doações: Todos, Pago, Pronto para Receber e Cancelado;
%     \item A Instituição informa os dados solicitados;
%     \item O Cliente envia os dados ao Sistema;
%     \item O Sistema valida os dados;
%     \item O Sistema retorna ao Cliente os dados da listagem requisitada;
%     \item O caso de uso se encerra.
%     \end{enumerate}
    
%     \textbf{Fluxo Alternativo A} \\
%     Em qualquer passo, caso exista algum erro de comunicação entre Cliente e Sistema:
%     \begin{enumerate}
%     \item Uma mensagem de erro é exibida ao Usuário informando da falha.
%     \end{enumerate}
    
%     \textbf{Fluxo Alternativo B} \\
%     No Passo 6, caso exista algum erro de validação das entradas:
%     \begin{enumerate}
%     \item Todos os erros de validação encontrados são retornados ao Cliente;
%     \item O fluxo retorna ao passo 2.
%     \end{enumerate}
  
  
  
%   \item \textbf{Dados de uma Doação}: Este fluxo especifica a ação de ver dados de uma única doação específica. Após escolher qual doação será exibida, a Instituição fornece os dados necessários ao cliente o qual, junto ao sistema, pega e exibe as informações requeridas.
%     \begin{itemize}
%     \item \textbf{Atores}: Instituição, Cliente;
%     \item \textbf{Pré-Condições}: Instituição cadastrada no Sistema, Instituição estar autenticada junto ao Sistema, Haverem Doações feitas a Instituição;
%     \item \textbf{Pós-Condições}: Nenhuma;
%     \item \textbf{Requisitos Funcionais}: O Sistema deve prover uma interface para expor dados de um pagamento.
%     \end{itemize}
	
%     \textbf{Fluxo Básico}
%     \begin{enumerate}
%     \item A Instituição decide ver dados de uma doação;
%     \item A Instituição seleciona qual doação será exibida;
%     \item A Instituição navega a uma página para exibição das informações da doação;
%     \item O Cliente requisita ao Sistema os dados da doação;
%     \item O Sistema busca os dados da doação especificada;
%     \item O Sistema retorna ao Cliente os dados da doação;
%     \item O caso de uso se encerra.
%     \end{enumerate}
    
%     \textbf{Fluxo Alternativo A} \\
%     Em qualquer passo, caso exista algum erro de comunicação entre Cliente e Sistema:
%     \begin{enumerate}
%     \item Uma mensagem de erro é exibida ao Usuário informando da falha.
%     \end{enumerate}
    
%     \textbf{Fluxo Alternativo B} \\
%     No Passo 5, caso a doação esteja marcada como Anônima (\codigo{anonymous = true}):
%     \begin{enumerate}
%     \item Todas as informações de identificação do doador são retiradas do objeto (\codigo{payeeName} e \codigo{payeeEmail});
%     \item O fluxo continua ao passo 6.
%     \end{enumerate}
% \end{lista}

%%%%%%%%%%%%%%%%%%%%%%%%%%%%%%%%%%%%%%%%%%%%%%%%%%%%%%%%%%%%%%%

