%% Resumo
\begin{resumo}

A internet e o e-mail revolucionaram a comunicação. Antes para enviar uma mensagem era necessário fazer isso pelos correios, um intermediário para fisicamente entrega-la. Retornar a esta realidade é inimaginável. O mesmo se deu com as Criptomoedas, que você pode transferir fundos de A para B em qualquer parte do mundo sem jamais precisar confiar em um terceiro para esta simples tarefa. 

Resolvendo problemas que antes inviabilizava à utilização de moedas digitais sem a necessidade de um intermediário, como o gasto duplo, o Bitcoin deu abertura a um leque de possibilidades impulsionando o desenvolvimento de novas tecnologias, serviços, aplicações e até mesmo outras criptomoedas. 

Uma prática comum entre negociantes de ações ou outros ativos financeiros é a formação de salas de  discussão e análise, sendo estas pagas ou não. Uma vez que o pregão deixou de ser presencial e tornou-se eletrônico as reuniões migraram para o meio virtual, seja para plataformas com este propósito seja para aplicativos de mensagens instantâneas. Da mesma forma, quem negocia criptomoedas também acompanha grupos com análises e até mesmo instruções para compra e venda. 

Assim, este trabalho apresenta o desenvolvimento de uma ferramenta destinada a administradores de grupos que compartilham instruções para compra e venda de criptomoedas, trazendo aos seus inscritos o benefício de não necessitarem realizar tais instruções de forma manual, mas sim automatizada pela presente ferramenta. Todas as interações por parte dos usuários com o sistema e vice-versa se dá através de um chatbot implementado utilizando a linguagem Python e destinada a plataforma de mensagens instantâneas Telegram.

\vspace{\onelineskip}
\noindent
\textbf{Palavras-chaves}: Chatbot, Criptomoedas, Bitcoin, Automação, Telegram.
\end{resumo}