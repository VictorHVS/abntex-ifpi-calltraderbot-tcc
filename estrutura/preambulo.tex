% Preambulo LaTeX: Define classes e características do documento
% Definição do docuemnto
\documentclass[
	%article,			% Define que este será um artigo (e não uma tese/monografia/relatório)
	12pt,				% Fonte: 12pt
	oneside,			% Impressão: oneside = 1 face, twoside = 2 faces (frente-e-verso)
    %openright,			% capítulos começam em página ímpar (use apenas se usar "twoside")
	a4paper,			% Tamanho do Papel: A4
    chapter=TITLE,		% Todos os capítulos devem ficam em caixa alta
    section=TITLE,		% Todas as seções devem ficar em caixa alta
	english,			% Adiciona Idioma para Hifenização: Inglês
    %spanish,			% Adiciona Idioma para Hifenização: Espanhol
    %french,			% Adiciona Idioma para Hifenização: Francês
	brazil				% Adiciona Idioma para Hifenização: Português do Brasil (o último idioma se torna o principal do documento)
]{abntex2}				% Utilizar ABNTeX2



%% Tipografia
%% Abra este arquivo e selecione uma das opções de fonte nele. A padrão é Times.
\input{configuracoes/tipografia}



%% Pacotes usados pelo documento (se não entender não mexa, hehehe)
\usepackage{courier}                    % Permite a utilização da fonte Courier (para códigos-fonte)
\usepackage[T1]{fontenc}				% Seleção de códigos de fonte.
\usepackage[utf8]{inputenc}				% Codificação do documento (conversão automática dos acentos)
\usepackage{indentfirst}				% Indenta o primeiro parágrafo de cada seção.
\usepackage{nomencl} 					% Usado pela Lista de símbolos
\usepackage{color}						% Controle das cores
\usepackage{graphicx}					% Inclusão de gráficos
\usepackage{float}						% Melhorias para posicionamento de gráficos e tabelas
\usepackage{microtype} 					% Melhorias na justificação
\usepackage{lastpage}   		        % Dá acesso ao número da última página do documento
\usepackage{booktabs}					% Comandos para tabelas
\usepackage{multirow, array}			% Múltiplas linhas e colunas em tabelas
\usepackage[hyphenbreaks]{breakurl}		% Hifenação para URLs no texto
\usepackage[table,xcdraw]{xcolor}       % Cores para algumas tabelas especiais
\usepackage[brazilian,hyperpageref]{backref}	 % Inclui nas Referências as páginas onde há as citações
\usepackage{simplecd}                   % Pacote para gerar capa do CD
\usepackage[final]{pdfpages}            % Pacote para incluir um PDF dentro de outro (ficha catalográfica)



%% Adiciona as alterações do ABNTeX-IFPI
\usepackage{abntex-ifpi/abntex-ifpi}	% Modificações do ABNTeX para o IFPI
\usepackage{abntex-ifpi/tikz-uml}	    % Pacote Tikz UML para criar UML no LaTeX



%% Metadados
\input{configuracoes/pdf}



%% Metadados
%% %%%%%%%%%%%%%%%%%%%%%%%%%%%%%%%%%%%%%%%%%%%%%%%% %%
%% Metadados do trabalho
%% AVISO: Todos esses dados serão automaticamente convertidos para caixa alta onde necessário
%% %%%%%%%%%%%%%%%%%%%%%%%%%%%%%%%%%%%%%%%%%%%%%%%% %%

%% Título
\titulo{CallTraderBot - Uma ferramenta de Chatbot para automação de compra e venda de criptomoedas no Telegram}

%% Autor
\autor{Victor Hugo Vieira de Sousa}

%% Nome do Curso (usado para a Capa do CD)
\nomedocurso{Análise e Desenvolvimento de Sistemas}

%% Local de publicação
\local{Teresina, Piauí}

%% Preâmbulo do trabalho
\preambulo{Projeto apresentado à Banca Examinadora como requisito para aprovação na disciplina de Trabalho de Conclusão de Curso II do Curso Superior de Tecnologia em Análise e Desenvolvimento de Sistemas do Instituto Federal de Educação, Ciência e Tecnologia do Piauí.}

%% Orientador
%% "M\textsuperscript{e}." = Abreviação oficial para "Mestre"
\orientador{Prof. Dr. Fábio de Jesus Lima Gomes}

%% Tipo de Trabalho
%% - Monografia
%% - Tese (Mestrado)
%% - Tese (Doutorado)
%% - Relatório técnico
\tipotrabalho{Monografia}

%% Data do Trabalho
\data{2018}

%% Nome da Instituição (para a capa)
\instituicao{INSTITUTO FEDERAL DE EDUCAÇÃO, CIÊNCIA E TECNOLOGIA DO PIAUÍ
\\
CAMPUS TERESINA CENTRAL
\\
TECNOLOGIA EM ANÁLISE E DESENVOLVIMENTO DE SISTEMAS}

%% Primeiro membro da banca examinadora
\membroum{Prof. M\textsuperscript{e}. Rogério da Silva}

%% Segundo membro da banca examinadora
\membrodois{Prof. M\textsuperscript{e}. Duany Dreyton Bezerra Sousa}

%% Terceiro membro da banca examinadora
%\membrotres{Prof. Dr. Ney Paranaguá de Carvalho}

%% Data da apresentação do trabalho
%% Se não souber a data da apresentação, utilize \underline{\hspace{3.5cm}}
%% Isso cria um sublinhado de 3.5cm, onde você pode escrever a data depois!
%\dataapresentacao{04 de Abril de 2017}
\dataapresentacao{\underline{\hspace{3.5cm}}}



%% Configuração do "Citado nas páginas"
\input{configuracoes/citacoes}



%% Cores
\input{configuracoes/cores}



%% Espaçamentos
\input{configuracoes/espacamentos}