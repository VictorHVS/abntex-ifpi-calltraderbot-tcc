%% %%%%%%%%%%%%%%%%%%%%%%%%%%%%%%%%%%%%%%%%%%%%%%%% %%
%% Metadados do trabalho
%% AVISO: Todos esses dados serão automaticamente convertidos para caixa alta onde necessário
%% %%%%%%%%%%%%%%%%%%%%%%%%%%%%%%%%%%%%%%%%%%%%%%%% %%

%% Título
\titulo{Trade Fight Club - Um simulador de mercado de ações}

%% Autor
\autor{Victor Hugo Vieira de Sousa}

%% Nome do Curso (usado para a Capa do CD)
\nomedocurso{Análise e Desenvolvimento de Sistemas}

%% Local de publicação
\local{Teresina, Piauí}

%% Preâmbulo do trabalho
\preambulo{Projeto apresentado à Banca Examinadora como requisito para aprovação na disciplina de Trabalho de Conclusão de Curso II do Curso Superior de Tecnologia em Análise e Desenvolvimento de Sistemas do Instituto Federal de Educação, Ciência e Tecnologia do Piauí.}

%% Orientador
%% "M\textsuperscript{e}." = Abreviação oficial para "Mestre"
\orientador{Prof. Dr. Fábio de Jesus Lima Gomes}

%% Tipo de Trabalho
%% - Monografia
%% - Tese (Mestrado)
%% - Tese (Doutorado)
%% - Relatório técnico
\tipotrabalho{Relatório técnico}

%% Data do Trabalho
\data{2020}

%% Nome da Instituição (para a capa)
\instituicao{INSTITUTO FEDERAL DE EDUCAÇÃO, CIÊNCIA E TECNOLOGIA DO PIAUÍ
\\
CAMPUS TERESINA CENTRAL
\\
TECNOLOGIA EM ANÁLISE E DESENVOLVIMENTO DE SISTEMAS}

%% Primeiro membro da banca examinadora
\membroum{Prof. M\textsuperscript{e}. TBD}

%% Segundo membro da banca examinadora
\membrodois{Prof. M\textsuperscript{e}. TBD}

%% Terceiro membro da banca examinadora
%\membrotres{Prof. Dr. Ney Paranaguá de Carvalho}

%% Data da apresentação do trabalho
%% Se não souber a data da apresentação, utilize \underline{\hspace{3.5cm}}
%% Isso cria um sublinhado de 3.5cm, onde você pode escrever a data depois!
%\dataapresentacao{04 de Abril de 2017}
\dataapresentacao{\underline{\hspace{3.5cm}}}